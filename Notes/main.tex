\documentclass[10pt, a4paper, onecolumn, openany]{book} % openany make chapter start whenever, DELETE in OFFICIAL

% PACKAGES:
% Font Encoding
\usepackage[utf8]{inputenc}     % Use UTF-8
\usepackage[T1]{fontenc}        % T1 font encoding (latin characters)
% Header
\usepackage{fancyhdr}           % fancy page header options
\usepackage{titlesec}           % used to have \thechapter in same line as \chaptertitlename
% MISC
\usepackage{hyperref}           % \url{}
\usepackage{xurl}
\usepackage{graphicx}           % images
\usepackage{xcolor}            % colors
\usepackage{fancyvrb}          % colors in Verbatin, header: \begin{Verbatim}[commandchars=\\\{\}]

% DECORATIVE LINES + CHAPTER IN SAME LINE:
\renewcommand{\headrulewidth}{2pt}  % Top decorative line
\renewcommand{\footrulewidth}{2pt}  % Bottom decorative line
\pagestyle{fancy}                   % better header for normal pages, not only chapter ones
\fancyhf{}                          % clear header and adjust as wanted:
    \chead{\leftmark}               % header
    \cfoot{Page \thepage}           % footer
\fancypagestyle{plain}{
\fancyhf{} 
    \chead{\leftmark}       % header
    \cfoot{Page \thepage}   % footer
}
\renewcommand{\chaptername}{}       % change word chapter to {}
\titleformat{\chapter}[hang]{\normalfont\huge\bfseries}{\chaptertitlename\ \thechapter.}{1em}{} % Chapter in same line as chapter name

% SIZES OF SECTIONS:
\titleformat*{\section}{\LARGE\bfseries}
\titleformat*{\subsection}{\Large\bfseries}
\titleformat*{\subsubsection}{\large\bfseries}

% DISABLE huge space after (paragraph indent) section name before text starts:
\setlength{\parindent}{0pt}

% COLOR EXAMPLES:
% \definecolor{MyColor}{RGB}{219, 48, 122}  % define
% \textcolor{MyColor}{Some random text}     % usage in document
\definecolor{root}{RGB}{222, 0, 0}
\definecolor{user}{RGB}{0, 150, 00}
\definecolor{dir}{RGB}{0, 100, 200}
\definecolor{file}{RGB}{77, 187, 101}
\definecolor{block}{RGB}{255, 80, 0}
\definecolor{command}{RGB}{41, 182, 0}
\definecolor{comment}{RGB}{0, 182, 182}
\definecolor{background}{RGB}{240, 240, 240}

% COLORS FOR CODE in document:
%\begin{minted}[frame=lines,framesep=2mm,baselinestretch=1.2,fontsize=\footnotesize,linenos]{js}
%\end{minted}

% IMAGES:
\graphicspath{./images/} % define directory
% \includegraphics[scale=1.5]{./images/random_image.png} % usage in document

% TABLE:
% \begin{center}
%    \begin{small}
%    \begin{tabular}{|p{1cm}|p{1cm}|p{1cm}|p{1cm}|p{1cm}|p{1cm}|p{1cm}|}
%    \hline
%    + & - & * & / & \% & ** & () \\
%    \hline
%    1 & 2 & 3 & 4 & 5 & 6 & 7 \\
%    \hline
%    \end{tabular}
%    \end{small}
%\end{center}

%\titlespacing*{\section}{0pt}{1.5cm}{0.2cm}
%\titlespacing*{\subsection}{0pt}{0.2cm}{0.2cm}

% colored verbatim BG:
%\let\oldv\verbatim
%\let\oldendv\endverbatim
%\def\verbatim{\par\setbox0\vbox\bgroup\oldv}
%\def\endverbatim{\oldendv\egroup\fboxsep0pt \noindent\colorbox[gray]{0.9}{\usebox0}\par}
% colored Verbatim BG (supported colors: https://linuxhint.com/change-text-colors-latex/):
\renewcommand{\FancyVerbFormatLine}[1]{\colorbox{background}{#1}}


%%%%%%%%%%%%%%%%%%%%%%%%%%%%%%%%%%%%%%%%%%%%%%%%%%%%%%%%%%%%%%%%%%%%%%%%%%%%
%%%%%%%%%%%%%%%%%%%%%%%%%%%%%%%%% TITLE %%%%%%%%%%%%%%%%%%%%%%%%%%%%%%%%%%%%
%%%%%%%%%%%%%%%%%%%%%%%%%%%%%%%%%%%%%%%%%%%%%%%%%%%%%%%%%%%%%%%%%%%%%%%%%%%%
\title{\textbf{Arch Linux}}
\author{AISK}
\date{January, 2022}
%%%%%%%%%%%%%%%%%%%%%%%%%%%%%%%%%%%%%%%%%%%%%%%%%%%%%%%%%%%%%%%%%%%%%%%%%%%%
%%%%%%%%%%%%%%%%%%%%%%%%%%%%%%%%% START %%%%%%%%%%%%%%%%%%%%%%%%%%%%%%%%%%%%
%%%%%%%%%%%%%%%%%%%%%%%%%%%%%%%%%%%%%%%%%%%%%%%%%%%%%%%%%%%%%%%%%%%%%%%%%%%%
\begin{document}
\maketitle
%\clearpage % official blank page
\tableofcontents

%%%%%%%%%%%%%%%%%%%%%%%%%%%%%%%%%%%%%%%%%%%%%%%%%%%%%%%%%%%%%%%%%%%%%%%%%%%%
%%%%%%%%%%%%%%%%%%%%%%%%%%%%%%%%% Flash USB %%%%%%%%%%%%%%%%%%%%%%%%%%%%%%%%%%%%
%%%%%%%%%%%%%%%%%%%%%%%%%%%%%%%%%%%%%%%%%%%%%%%%%%%%%%%%%%%%%%%%%%%%%%%%%%%%
\chapter{Flash USB}
%%%%%%%%%%%%%%%%%%%%%%%%%%%%%%%%%%%%%%%%%%%%%%%%%%%%%%%%%%%%%%%%%%%%%%%%%%%%
\section{Download Arch ISO}
\begin{enumerate}
    \item \textbf{Download Arch ISO from:}
\newline \underline{\href{https://archlinux.org/download/}{https://archlinux.org/download/}}
    \item \textbf{Verify Download:}
\begin{Verbatim}[commandchars=\\\{\}]
\textcolor{user}{user\$} \textcolor{command}{sha1sum} <\textcolor{file}{archlinux-YYYY.MM.DD-x86_64.iso}>
\end{Verbatim}
\end{enumerate}
%%%%%%%%%%%%%%%%%%%%%%%%%%%%%%%%%%%%%%%%%%%%%%%%%%%%%%%%%%%%%%%%%%%%%%%%%%%%
\section{USB Preparation}
\begin{enumerate}
    \item \textbf{Create Partition Table}
\begin{Verbatim}[commandchars=\\\{\}]
\textcolor{root}{root#} \textcolor{command}{parted} -s <\textcolor{block}{/dev/sdX}> mktable gpt
\end{Verbatim}
    \item \textbf{Print change:}
\begin{Verbatim}[commandchars=\\\{\}]
\textcolor{root}{root#} \textcolor{command}{parted} <\textcolor{block}{/dev/sdX}> (p)rint [free]
\end{Verbatim}
\end{enumerate}
%%%%%%%%%%%%%%%%%%%%%%%%%%%%%%%%%%%%%%%%%%%%%%%%%%%%%%%%%%%%%%%%%%%%%%%%%%%%
\section{Flash ISO to USB}
\begin{enumerate}
    \item \textbf{Unmount any mounted FS on HARD DRIVE!}
    \item \textbf{Flash to USB (/dev/sdX):}
\begin{Verbatim}[commandchars=\\\{\}]
\textcolor{root}{root#} \textcolor{command}{dd} if=<\textcolor{file}{./archlinux-YYYY.MM.DD-x86_64.iso}> of=<\textcolor{block}{/dev/sdX}> 
[bs=4M | status=progress]
\end{Verbatim}
\end{enumerate}
%%%%%%%%%%%%%%%%%%%%%%%%%%%%%%%%%%%%%%%%%%%%%%%%%%%%%%%%%%%%%%%%%%%%%%%%%%%%
\section{Boot Live Installer}
\subsection{Secure Boot}
Make sure, that Secure Boot is Disabled!
\begin{enumerate}
    \item \textbf{During POST press Key to access BIOS/UEFI:}
\newline \underline{\href{https://techofide.com/blogs/boot-menu-option-keys-for-all-computers-and-laptops-updated-list-2021-techofide/}{BIOS/UEFI Menu Keys For All Vendors}}
    \item \textbf{Disable Secure Boot}
    \item \textbf{Poweroff/Restart}
\end{enumerate}
\subsection{Boot}
\begin{enumerate}
    \item \textbf{Plug in Flashed USB}
    \item \textbf{During POST press Key to access Boot Menu:}
\newline \underline{\href{https://techofide.com/blogs/boot-menu-option-keys-for-all-computers-and-laptops-updated-list-2021-techofide/}{Boot Menu Keys For All Vendors}}
    \item \textbf{Select USB entry.}
\end{enumerate}
%%%%%%%%%%%%%%%%%%%%%%%%%%%%%%%%%%%%%%%%%%%%%%%%%%%%%%%%%%%%%%%%%%%%%%%%%%%%
%%%%%%%%%%%%%%%%%%%%%%%%%%%%%%%%% Pre-Installation %%%%%%%%%%%%%%%%%%%%%%%%%%%%%%%%%%%%
%%%%%%%%%%%%%%%%%%%%%%%%%%%%%%%%%%%%%%%%%%%%%%%%%%%%%%%%%%%%%%%%%%%%%%%%%%%%
\chapter{Pre-Installation}
%%%%%%%%%%%%%%%%%%%%%%%%%%%%%%%%%%%%%%%%%%%%%%%%%%%%%%%%%%%%%%%%%%%%%%%%%%%%
\section{Check Disk for bad sectors}
%%%%%%%%%%%%%%%%%%%%%%%%%%%%%%%%%%%%%%%%%%%%%%%%%%%%%%%%%%%%%%%%%%%%%%%%%%%%
\subsection{Theory}
\begin{itemize}
    \item \textbf{Block:} group of sectors, every file must occupy at least 1 block. 0b file occupy whole block.
    \begin{itemize}
        \item \textbf{512b =} good for lot of small files. More blocks = more metadata.
        \item \textbf{4096b =}  good for larger files, less metadata. Waste if there are small files.
    \end{itemize}
\end{itemize}
%%%%%%%%%%%%%%%%%%%%%%%%%%%%%%%%%%%%%%%%%%%%%%%%%%%%%%%%%%%%%%%%%%%%%%%%%%%%
\subsection{Disk Info gathering}
\label{disk_info_gathering}
\begin{itemize}
    \item \textbf{Find disks (block devices):}
\begin{Verbatim}[commandchars=\\\{\}]
\textcolor{user}{user\$} \textcolor{command}{lsblk} [-ap | -apf]
\textcolor{root}{root#} \textcolor{command}{fdisk -l} [\textcolor{block}{/dev/sdX}]
\textcolor{root}{root#} \textcolor{command}{blkid}
\end{Verbatim}
    \item \textbf{Get raw disk info:}
    \begin{itemize}
        \item \textbf{Disk size in bytes:}
\begin{Verbatim}[commandchars=\\\{\}]
\textcolor{root}{root#} \textcolor{command}{blockdev} [-v] --getsize64 <\textcolor{block}{/dev/sdX[Y]}>
\end{Verbatim}        
    \item \textbf{Disk block size in bytes:}
\begin{Verbatim}[commandchars=\\\{\}]
\textcolor{root}{root#} \textcolor{command}{blockdev} [-v] --getbsz <\textcolor{block}{/dev/sdX[Y]}>
\end{Verbatim}
    \item \textbf{Check if disk is readonly (1 = ro, 0 = rw):}
\begin{Verbatim}[commandchars=\\\{\}]
\textcolor{root}{root#} \textcolor{command}{blockdev} [-v] --getro <\textcolor{block}{/dev/sdX[Y]}>
\end{Verbatim}
    \end{itemize}
    \item \textbf{See partitions:}
\begin{Verbatim}[commandchars=\\\{\}]
\textcolor{root}{root#} \textcolor{command}{parted} <\textcolor{block}{/dev/sdX}> (p)rint [free]
\end{Verbatim}
\end{itemize}
%%%%%%%%%%%%%%%%%%%%%%%%%%%%%%%%%%%%%%%%%%%%%%%%%%%%%%%%%%%%%%%%%%%%%%%%%%%%
\subsection{Check Disk for bad sectors}
\begin{enumerate}
    \item \textbf{Unmount FS!}
    \item \textbf{Check disk for bad blocks:}
\begin{Verbatim}[commandchars=\\\{\}]
\textcolor{root}{root#} \textcolor{command}{badblocks} [-b 4096] [-w [-t 0xaa]] [-v] [-s] 
<\textcolor{block}{/dev/sdX[Y]}> | \textcolor{command}{tee} -a <\textcolor{file}{OUTPUT_FILE}>
\end{Verbatim}
\end{enumerate}

%%%%%%%%%%%%%%%%%%%%%%%%%%%%%%%%%%%%%%%%%%%%%%%%%%%%%%%%%%%%%%%%%%%%%%%%%%%%
%%%%%%%%%%%%%%%%%%%%%%%%%%%%%%%%% Installation %%%%%%%%%%%%%%%%%%%%%%%%%%%%%%%%%%%%
%%%%%%%%%%%%%%%%%%%%%%%%%%%%%%%%%%%%%%%%%%%%%%%%%%%%%%%%%%%%%%%%%%%%%%%%%%%%
\chapter{Installation}
%%%%%%%%%%%%%%%%%%%%%%%%%%%%%%%%%%%%%%%%%%%%%%%%%%%%%%%%%%%%%%%%%%%%%%%%%%%%
\section{ISO specific}
%%%%%%%%%%%%%%%%%%%%%%%%%%%%%%%%%%%%%%%%%%%%%%%%%%%%%%%%%%%%%%%%%%%%%%%%%%%%
\subsection{Remove pcspkr}
\begin{itemize}
    \item \textbf{Remove pcspkr module:}
\begin{Verbatim}[commandchars=\\\{\}]
\textcolor{root}{root#} \textcolor{command}{modprobe} -r pcspkr
\end{Verbatim}
\end{itemize}
%%%%%%%%%%%%%%%%%%%%%%%%%%%%%%%%%%%%%%%%%%%%%%%%%%%%%%%%%%%%%%%%%%%%%%%%%%%%
\subsection{Connect to WiFi}
\begin{enumerate}
    \item \textbf{Enable WiFi:}
\begin{Verbatim}[commandchars=\\\{\}]
\textcolor{root}{root#} \textcolor{command}{rfkill} unblock wlan
\end{Verbatim}
    \item \textbf{Start services:}
\begin{Verbatim}[commandchars=\\\{\}]
\textcolor{root}{root#} \textcolor{command}{systemctl} start wpa\_supplicant.service dhcpcd.service
\end{Verbatim}
    \item \textbf{Configure WiFi:}
\newline File (\textbf{\textcolor{file}{/etc/wpa\_supplicant/wpa\_supplicant.conf}}):
\begin{Verbatim}[commandchars=\\\{\}]
ctrl_interface=/run/wpa_supplicant
update_config=1
country=<2-LETTER-ISO-CODE>

\textcolor{comment}{# WPA-PSK protected:}
network=\{
    ssid="<ESSID>"
    scan_ssid=1 \textcolor{comment}{# Find hidden network}
    key_mgmt=WPA-PSK
    {psk="<PLAINTEXT-PASSWD>"}
    \textcolor{comment}{#psk=<32byte-HEX-NUMBER>}
    priority=1 \textcolor{comment}{# To which WiFi connect first}
\}

\textcolor{comment}{# WPA-EAP protected::}
network=\{
        ssid="<ESSID>"
        scan_ssid=1 \textcolor{comment}{# Find hidden network}
        key_mgmt=WPA-EAP
        \textcolor{comment}{#eap=PEAP}
        identity="<USERNAME>@<DOMAIN>"
        password="<PLAINTEXT-PASSWD>"
        \textcolor{comment}{#psk=<32byte-HEX-NUMBER>}
        \textcolor{comment}{#ca_cert="/etc/cert/ca.pem"}
        \textcolor{comment}{#phase1="peaplabel=0"}
        phase2="auth=MSCHAPV2"
        priority=2 \textcolor{comment}{# To which WiFi connect first}
\}

\textcolor{comment}{# Unprotected:}
network=\{
    ssid="<ESSID>"
    scan_ssid=1 \textcolor{comment}{# Find hidden network}
    key_mgmt=NONE
    priority=3 \textcolor{comment}{# To which WiFi connect first}
\}    
\end{Verbatim}

    
    
    \item \textbf{Connect to WiFi:}
\begin{Verbatim}[commandchars=\\\{\}]
\textcolor{root}{root#} \textcolor{command}{wpa_supplicant} -B -D wext -i <wlan0> 
-c <\textcolor{file}{/etc/wpa_supplicant/wpa_supplicant.conf}>
\end{Verbatim}
\end{enumerate}


%%%%%%%%%%%%%%%%%%%%%%%%%%%%%%%%%%%%%%%%%%%%%%%%%%%%%%%%%%%%%%%%%%%%%%%%%%%%
\section{Disk Partitioning}
%%%%%%%%%%%%%%%%%%%%%%%%%%%%%%%%%%%%%%%%%%%%%%%%%%%%%%%%%%%%%%%%%%%%%%%%%%%%
\subsection{GPT UEFI}
\begin{enumerate}
    \item \textbf{Get info about disks:}
\newline See section \underline{\textbf{\ref{disk_info_gathering}}}.

    \item \textbf{Create GPT Partition Table}
\begin{Verbatim}[commandchars=\\\{\}]
\textcolor{root}{root#} \textcolor{command}{parted} -s <\textcolor{block}{/dev/sdX}> mktable gpt
\end{Verbatim}

    \item \textbf{Create Paritions:}
    \begin{enumerate}
        \item \textbf{Enter cfdisk:}
\begin{Verbatim}[commandchars=\\\{\}]
\textcolor{root}{root#} \textcolor{command}{cfdisk}  <\textcolor{block}{/dev/sdX}>
\end{Verbatim}
        \item \textbf{Create EFI Partition (max 512MiB):}
\begin{Verbatim}[commandchars=\\\{\}]
\textcolor{root}{cfdisk>} \textcolor{command}{n}
\textcolor{root}{cfdisk>} \textcolor{command}{512MiB}
\textcolor{root}{cfdisk>} \textcolor{command}{t}
\textcolor{root}{cfdisk>} \textcolor{command}{EFI System}
\end{Verbatim}
        \item \textbf{Create Root Partition:}
\begin{Verbatim}[commandchars=\\\{\}]
\textcolor{root}{cfdisk>} \textcolor{command}{n}
\textcolor{root}{cfdisk>} \textcolor{command}{ } (Enter)
\end{Verbatim}        
        \item \textbf{Write Changes:}
\begin{Verbatim}[commandchars=\\\{\}]
\textcolor{root}{cfdisk>} \textcolor{command}{W}
\textcolor{root}{cfdisk>} \textcolor{command}{yes}
\end{Verbatim}
        \item \textbf{Quit cfdisk:}
\begin{Verbatim}[commandchars=\\\{\}]
\textcolor{root}{cfdisk>} \textcolor{command}{Q}
\end{Verbatim}
    \end{enumerate}

    \item \textbf{Create filesystems:}
    \begin{enumerate}
        \item \textbf{Create FAT32 for EFI:}
\begin{Verbatim}[commandchars=\\\{\}]
\textcolor{root}{root#} \textcolor{command}{mkfs.fat} [-F 32] [-n "EFI"] <\textcolor{block}{/dev/sdX1}>
\end{Verbatim}

        \item \textbf{Create Encrypted filesystem:}
        \begin{enumerate}
            \item \textbf{Encrypt root partition:}
\begin{Verbatim}[commandchars=\\\{\}]
\textcolor{root}{root#} \textcolor{command}{cryptsetup} [--label "LUKS"] luksFormat <\textcolor{block}{/dev/sdX2}>
> YES
> <PASSWORD>
> <PASSWORD (VERIFY)>
\end{Verbatim}
            \item \textbf{Open Encrypted root partition:}
\begin{Verbatim}[commandchars=\\\{\}]
\textcolor{root}{root#} \textcolor{command}{cryptsetup} open --type luks <\textcolor{block}{/dev/sdX2}> <luks_root>
> <PASSWORD>
\end{Verbatim}
        \item \textbf{Create EXT4 for Root:}
\begin{Verbatim}[commandchars=\\\{\}]
\textcolor{root}{root#} \textcolor{command}{mkfs.ext4} [-L "LUKS_ROOT"] <\textcolor{block}{/dev/mapper/luks_root}>
\end{Verbatim}      
        \end{enumerate}
    \end{enumerate}

    \item \textbf{OPTIONAL LUKS stuff:}
    \begin{itemize}
        \item \textbf{Close LUKS:}
\begin{Verbatim}[commandchars=\\\{\}]
\textcolor{root}{root#} \textcolor{command}{cryptsetup} close <luks_root>
\end{Verbatim}
\item \textbf{LUKS header:}
        \begin{enumerate}
            \item \textbf{See LUKS header:}
\begin{Verbatim}[commandchars=\\\{\}]
\textcolor{root}{root#} \textcolor{command}{cryptsetup} luksDump <\textcolor{block}{/dev/sdX2}>
\end{Verbatim}
            \item \textbf{Make LUKS header backup:}
\begin{Verbatim}[commandchars=\\\{\}]
\textcolor{root}{root#} \textcolor{command}{cryptsetup} luksHeaderBackup <\textcolor{block}{/dev/sdX2}>
--header-backup-file <\textcolor{file}{FILE}>
\end{Verbatim}
            \item \textbf{Destroy LUKS header :}
\begin{Verbatim}[commandchars=\\\{\}]
\textcolor{root}{root#} \textcolor{command}{cryptsetup} luksErase <\textcolor{block}{/dev/sdX2}>
\end{Verbatim}
            \item \textbf{restore LUKS header:}
\begin{Verbatim}[commandchars=\\\{\}]
\textcolor{root}{root#} \textcolor{command}{cryptsetup} luksHeaderRestore <\textcolor{block}{/dev/sdX2}> 
--header-backup-file <\textcolor{file}{FILE}>
\end{Verbatim}
        \end{enumerate}
    \end{itemize}

\end{enumerate}
%%%%%%%%%%%%%%%%%%%%%%%%%%%%%%%%%%%%%%%%%%%%%%%%%%%%%%%%%%%%%%%%%%%%%%%%%%%%
\section{Mount FS}
\label{mount_fs}
\begin{enumerate}
    \item \textbf{Mount Root filesystem:}
\begin{Verbatim}[commandchars=\\\{\}]
\textcolor{root}{root#} \textcolor{command}{mount} <\textcolor{block}{/dev/mapper/luks_root}> <\textcolor{dir}{/mnt/}>
\end{Verbatim}

    \item \textbf{Create boot dir:}
\begin{Verbatim}[commandchars=\\\{\}]
\textcolor{root}{root#} \textcolor{command}{mkdir} <\textcolor{dir}{/mnt/boot/}>
\end{Verbatim}
    \item \textbf{Mount boot partition:}
\begin{Verbatim}[commandchars=\\\{\}]
\textcolor{root}{root#} \textcolor{command}{mount} <\textcolor{block}{/dev/sdX1}> <\textcolor{dir}{/mnt/boot/}>
\end{Verbatim}
\end{enumerate}
%%%%%%%%%%%%%%%%%%%%%%%%%%%%%%%%%%%%%%%%%%%%%%%%%%%%%%%%%%%%%%%%%%%%%%%%%%%%
\section{Install Arch}
\begin{enumerate}
    \item \textbf{Check Mirrors:}
\begin{Verbatim}[commandchars=\\\{\}]
\textcolor{root}{root#} \textcolor{command}{cat} \textcolor{file}{/etc/pacman.d/mirrorlist}
\end{Verbatim}

    \item \textbf{Download Arch:}
\newline This installs BASE packages, LINUX kernel and common LINUX-FIRMWARE for common hardware:
\begin{Verbatim}[commandchars=\\\{\}]
\textcolor{root}{root#} \textcolor{command}{pacstrap} <\textcolor{dir}{/mnt/}> base linux [linux-firmware] 
\end{Verbatim}

    \item \textbf{Generate fstab:}
\begin{Verbatim}[commandchars=\\\{\}]
\textcolor{root}{root#} \textcolor{command}{genfstab} -U <\textcolor{dir}{/mnt/}> >> \textcolor{file}{/mnt/etc/fstab}
\end{Verbatim}

    \item \textbf{Chroot into arch:}
    \begin{enumerate}
        \item \textbf{Mount filesystems:}
\begin{Verbatim}[commandchars=\\\{\}]
\textcolor{root}{root#} \textcolor{command}{mount} -t proc \textcolor{dir}{/proc/} <\textcolor{dir}{/mnt/proc/}>
\textcolor{root}{root#} \textcolor{command}{mount} --rbind \textcolor{dir}{/sys/} <\textcolor{dir}{/mnt/sys/}>
\textcolor{root}{root#} \textcolor{command}{mount} --make-rslave <\textcolor{dir}{/mnt/sys/}>
\textcolor{root}{root#} \textcolor{command}{mount} --rbind \textcolor{dir}{/dev/} <\textcolor{dir}{/mnt/dev/}>
\textcolor{root}{root#} \textcolor{command}{mount} --make-rslave <\textcolor{dir}{/mnt/dev/}>
\end{Verbatim}
        \item \textbf{Chroot to root filesystem:}
\begin{Verbatim}[commandchars=\\\{\}]
\textcolor{root}{root#} \textcolor{command}{chroot} <\textcolor{dir}{/mnt/}> \textcolor{file}{/bin/bash}
\end{Verbatim}
    \end{enumerate}

    \item \textbf{Set up DNS for chrooted environment:}
\begin{Verbatim}[commandchars=\\\{\}]
\textcolor{root}{[root#]} \textcolor{command}{echo} "nameserver 1.1.1.1" > \textcolor{dir}{/etc/resolv.conf}
\end{Verbatim}

    \item \textbf{Install packages:}
    \begin{itemize}
        \item \textbf{Install vim:}
\begin{Verbatim}[commandchars=\\\{\}]
\textcolor{root}{[root#]} [\textcolor{command}{yes} |] \textcolor{command}{pacman} -S vim
\end{Verbatim}
    \end{itemize}

    \item \textbf{Add encrypted support to mkinitcpio:}
\newline File (\textbf{\textcolor{file}{/etc/mkinitcpio.conf}}):
\begin{Verbatim}[commandchars=\\\{\}]
...
HOOKS=(base udev autodetect modconf block \textcolor{file}{encrypt} filesystems
keyboard fsck)
...
\end{Verbatim}

    \item \textbf{Recreate initramfs with encrypted support:}
\begin{Verbatim}[commandchars=\\\{\}]
\textcolor{root}{[root#]} \textcolor{command}{mkinitcpio} -P
\end{Verbatim}
\end{enumerate}
%%%%%%%%%%%%%%%%%%%%%%%%%%%%%%%%%%%%%%%%%%%%%%%%%%%%%%%%%%%%%%%%%%%%%%%%%%%%
\section{Customize settings}
%%%%%%%%%%%%%%%%%%%%%%%%%%%%%%%%%%%%%%%%%%%%%%%%%%%%%%%%%%%%%%%%%%%%%%%%%%%%
\subsection{Time}
\begin{enumerate}
    \item \textbf{Select timezone:}
\begin{Verbatim}[commandchars=\\\{\}]
\textcolor{root}{[root#]} \textcolor{command}{ln} -sf <\textcolor{file}{/usr/share/zoneinfo/Europe/Copenhagen}> \textcolor{file}{/etc/localtime}
\end{Verbatim}
        \item \textbf{Update HW clock (generate: /etc/adjtime):}
\begin{Verbatim}[commandchars=\\\{\}]
\textcolor{root}{[root#]} \textcolor{command}{hwclock} --systohc
\end{Verbatim}
\end{enumerate}
%%%%%%%%%%%%%%%%%%%%%%%%%%%%%%%%%%%%%%%%%%%%%%%%%%%%%%%%%%%%%%%%%%%%%%%%%%%%
\subsection{Locales}
    \begin{enumerate}
        \item \textbf{Select locales:}
\newline File (\textbf{\textcolor{file}{/etc/locale.gen}}):
\begin{Verbatim}[commandchars=\\\{\}]
...
en_US.UTF-8 UTF-8
en_US ISO-8859-1
...
\end{Verbatim}
        \item \textbf{Generate locales:}
\begin{Verbatim}[commandchars=\\\{\}]
\textcolor{root}{[root#]} \textcolor{command}{locale-gen}
\end{Verbatim}      
        \item \textbf{Set language:}
\newline File (\textbf{\textcolor{file}{/etc/locale.conf}}):
\begin{Verbatim}[commandchars=\\\{\}]
LANG=en_US.UTF-8
\end{Verbatim}
        \item \textbf{Set keyboard:}
\newline File (\textbf{\textcolor{file}{/etc/vconsole.conf}}):
\begin{Verbatim}[commandchars=\\\{\}]
KEYMAP=us
\end{Verbatim}
        \end{enumerate}
%%%%%%%%%%%%%%%%%%%%%%%%%%%%%%%%%%%%%%%%%%%%%%%%%%%%%%%%%%%%%%%%%%%%%%%%%%%%
\subsection{Network}
        \begin{enumerate}
            \item \textbf{Set hostname:}
\newline File (\textbf{\textcolor{file}{/etc/hostname}}):
\begin{Verbatim}[commandchars=\\\{\}]
<HOSTNAME>
\end{Verbatim}
            \item \textbf{Install network packages:}
            \begin{itemize}
                \item \textbf{Install WiFi control:}
\begin{Verbatim}[commandchars=\\\{\}]
\textcolor{root}{[root#]} [\textcolor{command}{yes} |] \textcolor{command}{pacman} -S wpa_supplicant
\end{Verbatim}
                \item \textbf{Install DHCP client:}
\begin{Verbatim}[commandchars=\\\{\}]
\textcolor{root}{[root#]} [\textcolor{command}{yes} |] \textcolor{command}{pacman} -S dhcpcd
\end{Verbatim}
            \end{itemize}

    \end{enumerate}
%%%%%%%%%%%%%%%%%%%%%%%%%%%%%%%%%%%%%%%%%%%%%%%%%%%%%%%%%%%%%%%%%%%%%%%%%%%%
\section{Install bootloader}
\begin{enumerate}
    \item \textbf{Download packages:}
\begin{Verbatim}[commandchars=\\\{\}]
\textcolor{root}{[root#]} [\textcolor{command}{yes} |] \textcolor{command}{pacman} -S efibootmgr grub
\end{Verbatim}
    \item \textbf{Make sure EFI partition is mounted!}
\newline See section \underline{\textbf{\ref{mount_fs}}}.
    \item \textbf{Install GRUB:}
\begin{Verbatim}[commandchars=\\\{\}]
\textcolor{root}{[root#]} \textcolor{command}{grub-install} --target=x86_64-efi [--boot-directory=<\textcolor{dir}{/boot/}>]
--efi-directory=<\textcolor{dir}{/boot/}> --bootloader-id=<ArchLinux>
\end{Verbatim}

    \item \textbf{Find UUID of encrypted luks fs:}
\begin{Verbatim}[commandchars=\\\{\}]
\textcolor{root}{[root#]} \textcolor{command}{blkid} | \textcolor{command}{grep} "crypto_LUKS"
\end{Verbatim}    

    \item \textbf{Edit GRUB config for encryption:}
\newline File (\textbf{\textcolor{file}{/etc/default/grub}}):
\begin{Verbatim}[commandchars=\\\{\}]
GRUB_CMDLINE_LINUX="cryptdevice=UUID=<UUID>:<luks_root> \char92 
root=</dev/mapper/luks_root>"
\end{Verbatim}

    \item \textbf{Make/Update GRUB config file:}
\begin{Verbatim}[commandchars=\\\{\}]
\textcolor{root}{[root#]} \textcolor{command}{grub-mkconfig} -o \textcolor{file}{/boot/grub/grub.cfg}
\end{Verbatim}
\end{enumerate}
%%%%%%%%%%%%%%%%%%%%%%%%%%%%%%%%%%%%%%%%%%%%%%%%%%%%%%%%%%%%%%%%%%%%%%%%%%%%
\section{Finish installation}
%%%%%%%%%%%%%%%%%%%%%%%%%%%%%%%%%%%%%%%%%%%%%%%%%%%%%%%%%%%%%%%%%%%%%%%%%%%%
\subsection{Root Password}
\begin{enumerate}
    \item \textbf{Create root password:}
\begin{Verbatim}[commandchars=\\\{\}]
\textcolor{root}{[root#]} \textcolor{command}{passwd} root
> <PASSWORD>
> <PASSWORD-VERIFY>
\end{Verbatim}
\end{enumerate}
%%%%%%%%%%%%%%%%%%%%%%%%%%%%%%%%%%%%%%%%%%%%%%%%%%%%%%%%%%%%%%%%%%%%%%%%%%%%
\subsection{Finish installation}
\begin{enumerate}
    \item \textbf{Exit chroot:}
\begin{Verbatim}[commandchars=\\\{\}]
\textcolor{root}{[root#]} \textcolor{command}{exit}
\end{Verbatim}
    \item \textbf{Umount disk partitions:}
\begin{Verbatim}[commandchars=\\\{\}]
\textcolor{root}{root#} \textcolor{command}{umount} -R <\textcolor{dir}{/mnt/}>
\end{Verbatim}
    \item \textbf{Reboot:}
\begin{Verbatim}[commandchars=\\\{\}]
\textcolor{root}{root#} \textcolor{command}{poweroff}
\end{Verbatim}
\end{enumerate}
%%%%%%%%%%%%%%%%%%%%%%%%%%%%%%%%%%%%%%%%%%%%%%%%%%%%%%%%%%%%%%%%%%%%%%%%%%%%
%%%%%%%%%%%%%%%%%%%%%%%%%%%%%%%%% GRUB %%%%%%%%%%%%%%%%%%%%%%%%%%%%%%%%%%%%
%%%%%%%%%%%%%%%%%%%%%%%%%%%%%%%%%%%%%%%%%%%%%%%%%%%%%%%%%%%%%%%%%%%%%%%%%%%%
\chapter{GRUB}



%%%%%%%%%%%%%%%%%%%%%%%%%%%%%%%%%%%%%%%%%%%%%%%%%%%%%%%%%%%%%%%%%%%%%%%%%%%%
%%%%%%%%%%%%%%%%%%%%%%%%%%%%%%%%% References %%%%%%%%%%%%%%%%%%%%%%%%%%%%%%%%%%%%
%%%%%%%%%%%%%%%%%%%%%%%%%%%%%%%%%%%%%%%%%%%%%%%%%%%%%%%%%%%%%%%%%%%%%%%%%%%%
\chapter{References}
\begin{itemize}
    \item \textbf{Boot Procedure:}
\newline \underline{\href{https://wiki.archlinux.org/title/Arch\_boot\_process}{https://wiki.archlinux.org/title/Arch\_boot\_process}}
    \item \textbf{Partition Optimal:}
\newline \underline{\href{https://rainbow.chard.org/2013/01/30/how-to-align-partitions-for-best-performance-using-parted/}{Partitioning}}
    \item \textbf{Booted from UEFI:}
\begin{Verbatim}[commandchars=\\\{\}]
\textcolor{root}{root#} \textcolor{command}{ls /sys/firmware/efi/efivars}
\end{Verbatim}
\end{itemize}

%%%%%%%%%%%%%%%%%%%%%%%%%%%%%%%%%%%%%%%%%%%%%%%%%%%%%%%%%%%%%%%%%%%%%%%%%%%%
\href{https://wiki.archlinux.org/title/General_recommendations}{WIKI}



\end{document}
